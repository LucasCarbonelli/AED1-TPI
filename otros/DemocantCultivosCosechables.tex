\documentclass{article}
\usepackage[utf8]{inputenc}
\usepackage{listings}
\usepackage{color}

\definecolor{dkgreen}{rgb}{0,0.6,0}
\definecolor{gray}{rgb}{0.5,0.5,0.5}
\definecolor{mauve}{rgb}{0.58,0,0.82}

\lstset{frame=tb,
  language=C++,
  aboveskip=3mm,
  belowskip=3mm,
  showstringspaces=false,
  columns=flexible,
  basicstyle={\small\ttfamily},
  numbers=none,
  numberstyle=\tiny\color{gray},
  keywordstyle=\color{red},
  commentstyle=\color{blue},
  stringstyle=\color{mauve},
  breaklines=true,
  breakatwhitespace=true,
  tabsize=3
}

\title{Demostraciones TPI}
\author{juanmanuelbaldonado }
\date{June 2016}
\begin{document}

\maketitle

\section{listoParaCosechar}

\begin{lstlisting}

bool Sistema::listoParaCosechar() const
{
	return cantCultivosCosechables() >= 0.9 * ((campo().dimensiones().ancho * campo().dimensiones().largo) - 2);
}

int Sistema::cantCultivosCosechables() const{
	int cuenta = 0 ;
	int i = 0 ;

	// P_c: i == 0 ^ cuenta == 0 ^ this->campo().dimensiones().ancho > 0 ^ this->campo().dimensiones().largo > 0 (por invariante dimensionesValidas de Campo)

	while (i < this->campo().dimensiones().ancho){
		
		// I: 0 <= i <= this->campo().dimensiones().ancho ^ cuenta == |[1|k <- [0..i), j <- [0..this->campo().dimensiones().largo), this->campo().contenido((i,j)) == Cultivo ^ this->estadoDelCultivo((i,j)) == ListoParaCosechar ]|	
	
		cuenta = cuenta + contarFilasCosechables(i);

		i=i+1;
	}
	
	// Qc: i == this->campo().dimensiones().ancho ^ cuenta == |[1|k <- [0..this->campo().dimensiones().ancho), j <- [0..this->campo().dimensiones().largo), this->campo().contenido(p) == Cultivo ^ this->estadoDelCultivo(p) == ListoParaCosechar ]|
	
	return cuenta;
}



\end{lstlisting}







\maketitle

\section{P_c \Rightarrow I}

//vale  P_c: i == 0 \land cuenta == 0 \land this->campo().dimensiones().ancho > 0 \land this->campo().dimensiones().largo > 0 (por invariante dimensionesValidas de Campo)

\\ implica: i == 0 \land cuenta == |[1|k <- [0..i), j <- [0..this->campo().dimensiones().largo), this->campo().contenido((i,j)) == Cultivo \land this->estadoDelCultivo((i,j)) == ListoParaCosechar ]| == 0 \\
// implica: 0 <= i <= this->campo().dimensiones().ancho \land cuenta == 0 \\
// implica I \\
 
 
 
\maketitle

\section{(I \land \lnot B) \Rightarrow Q_c} 
 
// vale I \land !B \\
// implica: i == this->campo().dimensiones().ancho \land cuenta == |[1|k <- [0..i), j <- [0..this->campo().dimensiones().largo), this->campo().contenido((i,j)) == Cultivo \land this->estadoDelCultivo((i,j)) == ListoParaCosechar ]| \\
implica: i == this->campo().dimensiones().ancho \land cuenta == |[1|k <- [0..this->campo().dimensiones().ancho), j <- [0..this->campo().dimensiones().largo), this->campo().contenido((i,j)) == Cultivo \land this->estadoDelCultivo((i,j)) == ListoParaCosechar ]|

// implica Q_c \\


\maketitle

\section{ I \land f_v < cota -> !B}

// vale: I \land this->campo().dimensiones().ancho - i \leq 0 \\

En particular vale 0 \leq i \leq this->campo().dimensiones().ancho \land this->campo().dimensiones().ancho - i \leq 0 \\
// implica: i \leq this->campo().dimensiones().ancho \land this->campo().dimensiones().ancho \leq i \\
// implica: i == this->campo().dimensiones().ancho \\
// implica: \lnot B \\
 
 
 


\end{document}
